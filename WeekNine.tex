
\documentclass{ctexart}
%%%%%%%%%%%%%%%%%%%%%%%%%%%%%%%%%%%%%%%%%%%%%%%%%%%%%%%%%%%%%%%%%%%%%%%%%%%%%%%%%%%%%%%%%%%%%%%%%%%%%%%%%%%%%%%%%%%%%%%%%%%%%%%%%%%%%%%%%%%%%%%%%%%%%%%%%%%%%%%%%%%%%%%%%%%%%%%%%%%%%%%%%%%%%%%%%%%%%%%%%%%%%%%%%%%%%%%%%%%%%%%%%%%%%%%%%%%%%%%%%%%%%%%%%%%%
\usepackage{amsmath}


\setcounter{MaxMatrixCols}{10}
%TCIDATA{OutputFilter=LATEX.DLL}
%TCIDATA{Version=5.50.0.2953}
%TCIDATA{<META NAME="SaveForMode" CONTENT="1">}
%TCIDATA{BibliographyScheme=Manual}
%TCIDATA{Created=Saturday, November 14, 2015 15:32:15}
%TCIDATA{LastRevised=Friday, January 15, 2016 23:16:38}
%TCIDATA{<META NAME="GraphicsSave" CONTENT="32">}
%TCIDATA{<META NAME="DocumentShell" CONTENT="Standard LaTeX\Blank - Standard LaTeX Article">}
%TCIDATA{CSTFile=40 LaTeX article.cst}

\newtheorem{theorem}{Theorem}
\newtheorem{acknowledgement}[theorem]{Acknowledgement}
\newtheorem{algorithm}[theorem]{Algorithm}
\newtheorem{axiom}[theorem]{Axiom}
\newtheorem{case}[theorem]{Case}
\newtheorem{claim}[theorem]{Claim}
\newtheorem{conclusion}[theorem]{Conclusion}
\newtheorem{condition}[theorem]{Condition}
\newtheorem{conjecture}[theorem]{Conjecture}
\newtheorem{corollary}[theorem]{Corollary}
\newtheorem{criterion}[theorem]{Criterion}
\newtheorem{definition}[theorem]{Definition}
\newtheorem{example}[theorem]{Example}
\newtheorem{exercise}[theorem]{Exercise}
\newtheorem{lemma}[theorem]{Lemma}
\newtheorem{notation}[theorem]{Notation}
\newtheorem{problem}[theorem]{Problem}
\newtheorem{proposition}[theorem]{Proposition}
\newtheorem{remark}[theorem]{Remark}
\newtheorem{solution}[theorem]{Solution}
\newtheorem{summary}[theorem]{Summary}
\newenvironment{proof}[1][Proof]{\noindent\textbf{#1.} }{\ \rule{0.5em}{0.5em}}

\begin{document}



\bigskip \bigskip 统力第9周作业 \qquad 
赵丰\qquad 2013012178

5.2考虑分子内部能级时%
总粒子数$N=\underset{i}{\sum }\frac{\omega _{i}}{%
e^{\alpha +\beta \epsilon _{i}}-1},i$遍历所有的%
宏观态,$N$可分为两部分,

若$\epsilon _{i}\geq \epsilon _{1},$则此宏观%
态至少包括内部粒子%
处于激发态的情况$\implies N=%
\underset{\epsilon _{i}<\epsilon _{1}}{\sum }\frac{\omega _{i}}{e^{\alpha
+\beta \epsilon _{i}}-1}+\underset{\epsilon _{i}\geq \epsilon _{1}}{\sum }%
\frac{\omega _{i}}{e^{\alpha +\beta \epsilon _{i}}-1}$

$\underset{\epsilon _{i}\geq \epsilon _{1}}{\sum }\frac{\omega _{i}}{%
e^{\alpha +\beta \epsilon _{i}}-1}$ can be divided according to whether $%
\epsilon _{1}$ is taken into account.

$\implies N=N_{1}+N_{2},N_{1}$ 遍历所有基%
态的情况,$N_{2}$ 遍历所有%
激发态的情况

由准经典近似,$N_{1}=\int_{0}^{\infty
}g\left( \epsilon \right) \frac{d\epsilon }{e^{\alpha +\beta \epsilon }-1}%
,N_{2}=\int_{0}^{\infty }g\left( \epsilon \right) \frac{d\epsilon }{e^{\beta
\epsilon _{1}}e^{\alpha +\beta \epsilon }-1},$利用$g\left(
\epsilon \right) =CV\sqrt{\epsilon }$,

$C=2\pi \left( 2m\right) ^{3/2}h^{-3},\implies N=CV\left( \int_{0}^{\infty }%
\sqrt{\epsilon }\frac{d\epsilon }{e^{\alpha +\beta \epsilon }-1}%
+\int_{0}^{\infty }\sqrt{\epsilon }\frac{d\epsilon }{e^{\beta \epsilon
_{1}}e^{\alpha +\beta \epsilon }-1}\right) $

发生Bosen-Einstein凝聚时$\mu =0\implies
e^{\alpha }=e^{-\beta \mu }=1\implies $

$z$凝结温度和能量$\epsilon _{1}$%
的关系为$N=CV\left( \int_{0}^{\infty }\sqrt{\epsilon 
}\frac{d\epsilon }{e^{\beta \epsilon }-1}+\int_{0}^{\infty }\sqrt{\epsilon }%
\frac{d\epsilon }{e^{\beta \epsilon _{1}}e^{\beta \epsilon }-1}\right) $

其中$\beta =\frac{1}{kT_{c}},T_{c}$为凝结%
温度.

做变量替换$x=\frac{\epsilon }{kT_{c}}$得$%
N=CV\left( kT_{c}\right) ^{3/2}\left( \int_{0}^{\infty }\sqrt{x}\frac{dx}{%
e^{x}-1}+\int_{0}^{\infty }\sqrt{x}\frac{dx}{e^{\beta \epsilon _{1}}e^{x}-1}%
\right) $

若不考虑内部能级$N=CV\left(
kT_{c}^{0}\right) ^{3/2}\int_{0}^{\infty }\sqrt{x}\frac{dx}{e^{x}-1}\implies 
$

$\left( \frac{T_{c}^{0}}{T_{c}}\right) ^{3/2}=1+\frac{\int_{0}^{\infty }%
\sqrt{x}\frac{dx}{e^{\beta \epsilon _{1}}e^{x}-1}}{\int_{0}^{\infty }\sqrt{x}%
\frac{dx}{e^{x}-1}},$考虑到$\beta \epsilon _{1}=\frac{%
\epsilon }{kT_{c}}>>1,$

$\int_{0}^{\infty }\sqrt{x}\frac{dx}{e^{\beta \epsilon _{1}}e^{x}-1}\approx
\int_{0}^{\infty }\sqrt{x}\frac{dx}{e^{\beta \epsilon _{1}}e^{x}}=e^{-\beta
\epsilon _{1}}\Gamma \left( \frac{3}{2}\right) ,\implies $

$\frac{T_{c}}{T_{c}^{0}}=\left( 1+\frac{e^{-\beta \epsilon _{1}}\Gamma
\left( \frac{3}{2}\right) }{\int_{0}^{\infty }\sqrt{x}\frac{dx}{e^{x}-1}}%
\right) ^{-2/3}\approx 1-\frac{2}{3}\frac{e^{-\beta \epsilon _{1}}\Gamma
\left( \frac{3}{2}\right) }{\int_{0}^{\infty }\sqrt{x}\frac{dx}{e^{x}-1}},$%
数值计算给出$e^{-\beta \epsilon _{1}}$

系数为$-0.255.$\qquad 

$\int_{0}^{\frac{\pi }{2}}\sin \theta \cos \theta d\theta \int_{0}^{\infty
}v^{3}f\left( v\right) dv,$ the latter term is the collision number of gas
molecule per angle.


\end{document}
