
\documentclass{ctexart}
\usepackage{amsmath}


%%%%%%%%%%%%%%%%%%%%%%%%%%%%%%%%%%%%%%%%%%%%%%%%%%%%%%%%%%%%%%%%%%%%%%%%%%%%%%%%%%%%%%%%%%%%%%%%%%%%%%%%%%%%%%%%%%%%%%%%%%%%%%%%%%%%%%%%%%%%%%%%%%%%%%%%%%%%%%%%%%%%%%%%%%%%%%%%%%%%%%%%%%%%%%%%%%%%%%%%%%%%%%%%%%%%%%%%%%%%%%%%%%%%
%TCIDATA{OutputFilter=LATEX.DLL}
%TCIDATA{Version=5.00.0.2552}
%TCIDATA{<META NAME="SaveForMode" CONTENT="1">}
%TCIDATA{Created=Saturday, November 28, 2015 12:10:24}
%TCIDATA{LastRevised=Saturday, November 28, 2015 16:44:02}
%TCIDATA{<META NAME="GraphicsSave" CONTENT="32">}
%TCIDATA{<META NAME="DocumentShell" CONTENT="Scientific Notebook\Blank Document">}
%TCIDATA{CSTFile=Math with theorems suppressed.cst}
%TCIDATA{PageSetup=72,72,72,72,0}
%TCIDATA{AllPages=
%F=36,\PARA{038<p type="texpara" tag="Body Text" >\hfill \thepage}
%}


\newtheorem{theorem}{Theorem}
\newtheorem{acknowledgement}[theorem]{Acknowledgement}
\newtheorem{algorithm}[theorem]{Algorithm}
\newtheorem{axiom}[theorem]{Axiom}
\newtheorem{case}[theorem]{Case}
\newtheorem{claim}[theorem]{Claim}
\newtheorem{conclusion}[theorem]{Conclusion}
\newtheorem{condition}[theorem]{Condition}
\newtheorem{conjecture}[theorem]{Conjecture}
\newtheorem{corollary}[theorem]{Corollary}
\newtheorem{criterion}[theorem]{Criterion}
\newtheorem{definition}[theorem]{Definition}
\newtheorem{example}[theorem]{Example}
\newtheorem{exercise}[theorem]{Exercise}
\newtheorem{lemma}[theorem]{Lemma}
\newtheorem{notation}[theorem]{Notation}
\newtheorem{problem}[theorem]{Problem}
\newtheorem{proposition}[theorem]{Proposition}
\newtheorem{remark}[theorem]{Remark}
\newtheorem{solution}[theorem]{Solution}
\newtheorem{summary}[theorem]{Summary}
\newenvironment{proof}[1][Proof]{\noindent\textbf{#1.} }{\ \rule{0.5em}{0.5em}}


\begin{document}



统力第11周作业\qquad 赵丰%
2013012178\bigskip 

5.8绝对零度下自由电子%
在$0\symbol{126}\epsilon _{F}$能级上每个%
量子态上粒子数恰为1%
而$\epsilon _{F}$能级以上无粒%
子分布,对二维费米气,%
准经典近似下能态密%
度$g\left( \epsilon \right) d\epsilon =\frac{V}{h^{2}}2\pi g_{s}pdp=%
\frac{2\pi mVg_{s}}{h^{2}}d\epsilon $

$N=\int_{0}^{\epsilon _{F}}g\left( \epsilon \right) d\epsilon =\frac{2\pi
mVg_{s}}{h^{2}}\epsilon _{F}\implies $Fermi energy $\epsilon _{F}=\frac{%
h^{2}N}{2\pi mVg_{s}}.$

$N\overline{\epsilon }_{0}=\int_{0}^{\epsilon _{F}}\epsilon g\left( \epsilon
\right) d\epsilon =\frac{2\pi mVg_{s}}{h^{2}}\frac{\epsilon _{F}^{2}}{2}%
\implies \overline{\epsilon }_{0}=\frac{2\pi mVg_{s}}{h^{2}N}\frac{\epsilon
_{F}^{2}}{2}=\frac{\epsilon _{F}}{2}.$

5.9绝对零度下相对论粒%
子在$0\symbol{126}\epsilon _{F}$能级上每%
个量子态上粒子数恰%
为1而$\epsilon _{F}$能级以上无%
粒子分布,准连续近似%
下能态密度$g\left( \epsilon \right) d\epsilon =%
\frac{g_{s}V}{h^{3}}4\pi p^{2}dp=\frac{4\pi g_{s}V}{h^{3}c^{3}}\epsilon
^{2}d\epsilon $

$N=\int_{0}^{\epsilon _{F}}g\left( \epsilon \right) d\epsilon =\frac{4\pi
g_{s}V}{3h^{3}c^{3}}\epsilon _{F}^{3}\implies \epsilon _{F}=\sqrt[3]{\frac{%
3h^{3}c^{3}N}{4\pi g_{s}V}}$

考虑到电子自旋简并%
度为2$\implies $Fermi energy $\epsilon _{F}=hc\sqrt[3]{\frac{3N}{%
8\pi g_{s}V}}.$

$N\overline{E}=\int_{0}^{\epsilon _{F}}\epsilon g\left( \epsilon \right)
d\epsilon =\frac{\pi g_{s}V\epsilon _{F}^{4}}{h^{3}c^{3}}\implies \overline{E%
}=\frac{\pi g_{s}V\epsilon _{F}^{4}}{h^{3}c^{3}N}=\frac{3\epsilon _{F}}{4}.$

5.10方均根速率即为平均%
能量时的速率:仿照前%
两题可求先得$\overline{E}=\frac{3}{5}%
\epsilon _{F}\implies $ $\overline{v^{2}}=\frac{3}{5}v_{F}^{2},$速%
率分布的概率密度函%
数为: $f\left( v\right) dv=\frac{g\left( \epsilon \right)
d\epsilon }{N}=\frac{3\epsilon d\epsilon }{2\epsilon _{F}^{3/2}}=\frac{%
3v^{2}dv}{v_{F}^{3}}$

$\overline{v}=\int_{0}^{v_{F}}vf\left( v\right) dv=\int_{0}^{v_{F}}\frac{%
3v^{3}dv}{v_{F}^{3}}=\frac{3}{4}v_{F}.$

$\overline{v}_{x}=\underset{v_{x}^{2}+v_{y}^{2}+v_{z}^{2}\leq \frac{%
2\epsilon _{F}}{m}}{\int }v_{x}f\left( v\right) dv_{x}=0,$

$\overline{v}_{x}^{2}=\underset{v_{x}^{2}+v_{y}^{2}+v_{z}^{2}\leq \frac{%
2\epsilon _{F}}{m}}{\int }v_{x}^{2}f\left( v\right) dv_{x}=2\int_{0}^{\sqrt{%
\frac{2\epsilon _{F}}{m}-v_{y}^{2}-v_{z}^{2}}}v_{x}^{2}f\left( v\right)
dv_{x}$

$=\frac{6}{v_{F}^{3}}\left( \frac{\left( \frac{2\epsilon _{F}}{m}%
-v_{y}^{2}-v_{z}^{2}\right) ^{5/2}}{5}+\left( v_{y}^{2}+v_{z}^{2}\right) 
\frac{\left( \frac{2\epsilon _{F}}{m}-v_{y}^{2}-v_{z}^{2}\right) ^{3/2}}{3}%
\right) ,$

5.11 由5.9知,此种情况$g\left( \epsilon
\right) =\frac{8\pi V}{h^{3}c^{3}}\epsilon ^{2}=C\epsilon ^{2},T=0K\implies
N=\frac{C}{3}\epsilon _{F}^{3},$

$\implies \mu _{0}=\epsilon _{F}=hc\sqrt[3]{\frac{3N}{8\pi V}}$

由粒子数守恒,低温条%
件下同样有$N=\int_{0}^{\infty }g\left(
\epsilon \right) \frac{1}{e^{\alpha +\beta \epsilon }+1}d\epsilon =\frac{C}{3%
}\epsilon _{F}^{3}$

$\implies \int_{0}^{\infty }\frac{\epsilon ^{2}}{e^{\alpha +\beta \epsilon
}+1}d\epsilon =\frac{1}{3}\epsilon _{F}^{3},$

$\int_{0}^{\infty }\frac{\epsilon ^{2}}{e^{\alpha +\beta \epsilon }+1}%
d\epsilon =$分部积分一次有$=\frac{%
\beta }{3}\int_{0}^{\infty }\epsilon ^{3}\frac{e^{\alpha +\beta \epsilon }}{%
\left( e^{\alpha +\beta \epsilon }+1\right) ^{2}}d\epsilon =\frac{\beta }{3}%
\int_{0}^{\infty }\left( \epsilon -\mu +\mu \right) ^{3}\frac{e^{\beta
\left( \epsilon -\mu \right) }}{\left( e^{\beta \left( \epsilon -\mu \right)
}+1\right) ^{2}}d\epsilon ,$做变量替换$z=\beta
\left( \epsilon -\mu \right) ,$

$=\frac{1}{3\beta ^{3}}\int_{-\beta \mu }^{\infty }\left( z+\beta \mu
\right) ^{3}\frac{e^{z}}{\left( e^{\beta z}+1\right) ^{2}}dz,$低%
温条件下$\beta \mu \rightarrow \infty ,$上%
式可近似为

$=\frac{1}{3\beta ^{3}}\int_{-\infty }^{\infty }\left( z+\beta \mu \right)
^{3}\frac{e^{z}}{\left( e^{z}+1\right) ^{2}}dz$考虑到%
函数$\frac{e^{z}}{\left( e^{z}+1\right) ^{2}}$为偶%
函数,将$\left( z+\beta \mu \right) ^{3}$展开%
后只有$z$的偶次项有贡%
献:

$=\frac{1}{3\beta ^{3}}\left( \int_{-\infty }^{\infty }\beta ^{3}\mu ^{3}%
\frac{e^{z}}{\left( e^{z}+1\right) ^{2}}dz+3\beta \mu \int_{-\infty
}^{\infty }z^{2}\frac{e^{z}}{\left( e^{z}+1\right) ^{2}}dz\right) $

$=\frac{\mu ^{3}}{3}+\frac{\mu }{\beta ^{2}}\pi ^{2}$

$\implies $ $\frac{1}{3}\mu _{0}^{3}=\frac{\mu ^{3}}{3}+\frac{\mu }{\beta
^{2}}\pi ^{2},$ can be solved by perturbation theory, 

取初值$\mu =\mu _{0},$代入迭代%
方程$\mu _{n+1}=\sqrt[3]{\mu _{0}^{3}-\frac{\pi ^{2}\mu _{n}}{%
\beta ^{2}}}$得到$\mu $的近似解为

$\sqrt[3]{\mu _{0}^{3}-\frac{\pi ^{2}\mu _{0}}{\beta ^{2}}}=\mu _{0}\left( 1-%
\frac{\pi ^{2}}{\beta ^{2}\mu _{0}^{2}}\right) ^{1/3}=\mu _{0}\left( 1-\frac{%
\pi ^{2}k^{2}T^{2}}{3\mu _{0}^{2}}\right) .$

$\overline{E}=\int_{0}^{\infty }\frac{g\left( \epsilon \right) \epsilon }{%
e^{\alpha +\beta \epsilon }+1}d\epsilon =C\int_{0}^{\infty }\frac{\epsilon
^{3}}{e^{\alpha +\beta \epsilon }+1}d\epsilon $类似上%
面的方法可求出

$\int_{0}^{\infty }\frac{\epsilon ^{3}}{e^{\alpha +\beta \epsilon }+1}%
d\epsilon =\frac{1}{4\beta ^{4}}\int_{-\infty }^{\infty }\left( z+\beta \mu
\right) ^{4}\frac{e^{z}}{\left( e^{z}+1\right) ^{2}}dz$

$=\frac{1}{4\beta ^{4}}\left[ \int_{-\infty }^{\infty }\left( \beta \mu
\right) ^{4}\frac{e^{z}}{\left( e^{z}+1\right) ^{2}}dz+\int_{-\infty
}^{\infty }6\left( \beta \mu \right) ^{2}\frac{z^{2}e^{z}}{\left(
e^{z}+1\right) ^{2}}dz+\int_{-\infty }^{\infty }\frac{z^{4}e^{z}}{\left(
e^{z}+1\right) ^{2}}dz\right] $

=$\frac{1}{4}\mu ^{4}+\frac{\mu ^{2}\pi ^{2}}{2\beta ^{2}}+O\left( \frac{1}{%
\beta ^{4}}\right) $

$\overline{E}$可近似为$C\left[ \frac{1}{4}\mu ^{4}+%
\frac{\mu ^{2}\pi ^{2}k^{2}T^{2}}{2}\right] ,$

$\frac{1}{4}\mu ^{4}+\frac{\mu ^{2}\pi ^{2}k^{2}T^{2}}{2}=\frac{1}{4}\mu
_{0}^{4}\left( 1-\frac{\pi ^{2}k^{2}T^{2}}{3\mu _{0}^{2}}\right) ^{4}+\frac{%
\pi ^{2}k^{2}T^{2}\mu _{0}^{2}}{2}\left( 1-\frac{\pi ^{2}k^{2}T^{2}}{3\mu
_{0}^{2}}\right) ^{2}$

$=\frac{1}{4}\mu _{0}^{4}\left( 1-\frac{4\pi ^{2}k^{2}T^{2}}{3\mu _{0}^{2}}%
\right) +\frac{\pi ^{2}k^{2}T^{2}\mu _{0}^{2}}{2}\left( 1-\frac{2\pi
^{2}k^{2}T^{2}}{3\mu _{0}^{2}}\right) $

$=C^{\prime }+\frac{\pi ^{2}k^{2}T^{2}\mu _{0}^{2}}{6}-O\left( T^{4}\right) $

$C_{v}=\frac{d\overline{E}}{dT}\approx \frac{8\pi V}{h^{3}c^{3}}\frac{\pi
^{2}k^{2}T\mu _{0}^{2}}{3}=\frac{8\pi ^{3}k^{2}\mu _{0}^{2}V}{3h^{3}c^{3}}T,$%
代入$\mu _{0}=hc\sqrt[3]{\frac{3N}{8\pi V}}$

$C_{v}=\frac{\pi ^{2}k^{2}}{\mu _{0}}NT.$


\end{document}
