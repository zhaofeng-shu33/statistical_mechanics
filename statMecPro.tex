\documentclass[10pt]{article}

\begin{document}

\setlength{\parindent}{2em}
\par
问题:生n,每生从四师,以保证公平也。然生之也无限,师之精力有限,即n可任意大,而每师可带之弟子门徒,则有一上限也,问师之数量m可有最小值乎?
\par
为精确描述“师之精力有限”,又有人作如下合理假定:\\
   “任意两个学生最多共一师”,求m之最小。\\
   生1,每师带1徒,共需4师;\\
   每师带2徒,各师工作量皆相等,而生两两恰共1师,此真为理想之情况,先解之以抛砖引玉:生两两恰共1师,即:
$ {n \choose 2}=m \quad and \quad 4\times n-{n \choose 2}=m $
\\
\(g(\epsilon)=2\)\\
\(g(\epsilon)=\frac{4\pi V(2m)^\frac{3}{2}}{h^3}\)
%comment
\(A=\frac{C m^\frac{3}{2}}{4 \sqrt{2} \pi}\)
\( \overline{v}=\frac{3}{2m^2\beta^2 v_{_F}^3}\)
(1)\(g(\epsilon)=\frac{4\pi(2m)^\frac{3}{2}V}{h^3}\)

\end{document}