
\documentclass{ctexart}
%%%%%%%%%%%%%%%%%%%%%%%%%%%%%%%%%%%%%%%%%%%%%%%%%%%%%%%%%%%%%%%%%%%%%%%%%%%%%%%%%%%%%%%%%%%%%%%%%%%%%%%%%%%%%%%%%%%%%%%%%%%%%%%%%%%%%%%%%%%%%%%%%%%%%%%%%%%%%%%%%%%%%%%%%%%%%%%%%%%%%%%%%%%%%%%%%%%%%%%%%%%%%%%%%%%%%%%%%%%%%%%%%%%%%%%%%%%%%%%%%%%%%%%%%%%%
\usepackage{amsmath}
\usepackage{amssymb}

\setcounter{MaxMatrixCols}{10}
%TCIDATA{OutputFilter=LATEX.DLL}
%TCIDATA{Version=5.00.0.2552}
%TCIDATA{<META NAME="SaveForMode" CONTENT="1">}
%TCIDATA{Created=Friday, October 09, 2015 07:37:58}
%TCIDATA{LastRevised=Saturday, October 31, 2015 23:56:04}
%TCIDATA{<META NAME="GraphicsSave" CONTENT="32">}
%TCIDATA{<META NAME="DocumentShell" CONTENT="Scientific Notebook\Blank Document">}
%TCIDATA{CSTFile=Math with theorems suppressed.cst}
%TCIDATA{PageSetup=72,72,72,72,0}
%TCIDATA{AllPages=
%F=36,\PARA{038<p type="texpara" tag="Body Text" >\hfill \thepage}
%}


\newtheorem{theorem}{Theorem}
\newtheorem{acknowledgement}[theorem]{Acknowledgement}
\newtheorem{algorithm}[theorem]{Algorithm}
\newtheorem{axiom}[theorem]{Axiom}
\newtheorem{case}[theorem]{Case}
\newtheorem{claim}[theorem]{Claim}
\newtheorem{conclusion}[theorem]{Conclusion}
\newtheorem{condition}[theorem]{Condition}
\newtheorem{conjecture}[theorem]{Conjecture}
\newtheorem{corollary}[theorem]{Corollary}
\newtheorem{criterion}[theorem]{Criterion}
\newtheorem{definition}[theorem]{Definition}
\newtheorem{example}[theorem]{Example}
\newtheorem{exercise}[theorem]{Exercise}
\newtheorem{lemma}[theorem]{Lemma}
\newtheorem{notation}[theorem]{Notation}
\newtheorem{problem}[theorem]{Problem}
\newtheorem{proposition}[theorem]{Proposition}
\newtheorem{remark}[theorem]{Remark}
\newtheorem{solution}[theorem]{Solution}
\newtheorem{summary}[theorem]{Summary}
\newenvironment{proof}[1][Proof]{\noindent\textbf{#1.} }{\ \rule{0.5em}{0.5em}}
\def\NEG#1{\mathbb{#1}}

\begin{document}



\bigskip 统力第四周作业 \qquad 
赵丰\qquad 2013012178

\bigskip 题目:试计算理想气%
体的熵

解答:方法一\qquad 使用准%
经典近似的微正则分%
布:

在能量为E附近,分子数%
为N的理想气体微观状%
态数为$\Omega \left( E\right) =\frac{V^{N}}{N!h^{3N}\Gamma
\left( \frac{3N}{2}+1\right) }\frac{3N}{2}\left( 2\pi m\right) ^{\frac{3N}{2}%
}E^{\frac{3N}{2}-1}dE$

$\left( \text{上次作业题已算%
过,其中用到了等概率%
假设,因为粒子不可分%
辨,所以要除以}N\text{!}\right) $

其中$dE$表示能量壳层,$dE$%
前面的项为概率密度$,$%
是从系统分布函数即%
能量小于E的状态数求%
导得到的$,$出现$N!$是考%
虑到微观粒子的全同%
性$,$而$\left( h^{3}\right) ^{N}$则是使用%
准经典的近似时在3N维%
的相空间一个状态所%
占据的体积$\frac{\pi ^{\frac{3N}{2}}}{\Gamma
\left( \frac{3N}{2}+1\right) }$则对应3N维球%
体的体积$,$由此式代入%
熵的定义式$,$可得能量%
为E时系统的熵$S\left( E\right) =k\ln
\Omega \left( E\right) $

$\ln \Omega \left( E\right) =\ln \left( \frac{V^{N}}{N!h^{3N}\Gamma \left( 
\frac{3N}{2}+1\right) }\frac{3N}{2}\left( 2\pi m\right) ^{\frac{3N}{2}}E^{%
\frac{3N}{2}-1}dE\right) =N\ln \left( \frac{V\left( 2\pi mE\right) ^{\frac{3%
}{2}}}{h^{3}}\right) +\ln \left( \frac{3NdE}{2E}\right) +\ln \left( \frac{1}{%
N!\Gamma \left( \frac{3N}{2}+1\right) }\right) ;$

$\;$为计算最后一项$,$使%
用$\Gamma \left( \frac{3N}{2}+1\right) \approx \frac{3N}{2}!$的%
近似,则由Sterling's formula $\ln n!\approx n\ln n-n$
if n is very large.

$\ln \left( \frac{1}{N!\Gamma \left( \frac{3N}{2}+1\right) }\right) =-\ln
\left( \frac{3N}{2}!\right) -\ln \left( N!\right) =\frac{3N}{2}-\frac{3N}{2}%
\ln \left( \frac{3N}{2}\right) +N-N\ln N$

=$\frac{5}{2}N+N\ln \frac{1}{N\cdot \left( \frac{3}{2}N\right) ^{3/2}}$%
代入$\ln \Omega \left( E\right) $中有

$\ln \Omega \left( E\right) =N\ln \left( \frac{V\left( 2\pi mE\right) ^{%
\frac{3}{2}}}{h^{3}}\right) +\ln \left( \frac{3NdE}{2E}\right) +\frac{5}{2}%
N+N\ln \frac{1}{N\cdot \left( \frac{3}{2}N\right) ^{3/2}}$

$=N\ln \left( \frac{V}{h^{3}N}\left( \frac{4\pi mE}{3N}\right) ^{3/2}\right)
+\ln \left( \frac{3NdE}{2E}\right) +\frac{5}{2}N$

其中$\ln \left( \frac{3NdE}{2E}\right) $相比其%
它两项可以忽略$\implies $

$S\left( E\right) =k\ln \Omega \left( E\right) =Nk\ln \left( \frac{V}{h^{3}N}%
\left( \frac{4\pi mE}{3N}\right) ^{3/2}\right) +\frac{5}{2}Nk$

\bigskip 方法二\qquad 使用准经%
典的正则分布

先求系统的Partition Function $Z=\frac{1}{%
N!h^{3N}}\int e^{-\beta H\left( \vec{q},\vec{p}\right) }d\Omega =\frac{1}{%
N!h^{3N}}\underset{\vec{q},\vec{p}\in \NEG{R}^{3n}}{\idotsint }e^{-\beta 
\underset{i=1}{\overset{3n}{\Sigma }}\frac{p_{i}^{2}}{2m}}d\Omega $

=$\frac{V^{N}}{N!h^{3N}}\left( \int_{-\infty }^{\infty }e^{-\beta \frac{%
p_{i}^{2}}{2m}}dp_{i}\right) ^{3n}\overset{Gaussian\_Integral}{=}\frac{V^{N}%
}{N!h^{3N}}\left( \frac{2\pi m}{\beta }\right) ^{3N/2}.$

由正则分布的热力学%
公式\qquad $E=-\left( \frac{\partial \ln Z}{\partial \beta }%
\right) _{N,V}=\frac{3}{2}N\frac{1}{\beta }=\frac{3}{2}NkT.$

$S=k\left( \ln Z+\beta E\right) =k\left( N\ln \left[ \frac{V}{h^{3}}\left( 
\frac{2\pi m}{\beta }\right) ^{3/2}\right] -\ln N!+\frac{3}{2}N\right) $

由Stirling's formula $S=k\left( N\ln \left[ \frac{V}{h^{3}}\left( 
\frac{2\pi m}{\beta }\right) ^{3/2}\right] -N\left( \ln \left( N\right)
-1\right) +\frac{3}{2}N\right) $

$=kN\ln \left[ \frac{V}{h^{3}N}\left( \frac{2\pi m}{\beta }\right) ^{3/2}%
\right] $+$\frac{5}{2}kN$

将$\beta $用能量代换即得%
到和方法一相同的结%
果

$S=kN\ln \left[ \frac{V}{h^{3}N}\left( \frac{4\pi mE}{3N}\right) ^{3/2}%
\right] $+$\frac{5}{2}kN$

\bigskip


\end{document}
