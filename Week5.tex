
\documentclass{ctexart}
%%%%%%%%%%%%%%%%%%%%%%%%%%%%%%%%%%%%%%%%%%%%%%%%%%%%%%%%%%%%%%%%%%%%%%%%%%%%%%%%%%%%%%%%%%%%%%%%%%%%%%%%%%%%%%%%%%%%%%%%%%%%%%%%%%%%%%%%%%%%%%%%%%%%%%%%%%%%%%%%%%%%%%%%%%%%%%%%%%%%%%%%%%%%%%%%%%%%%%%%%%%%%%%%%%%%%%%%%%%%%%%%%%%%%%%%%%%%%%%%%%%%%%%%%%%%
\usepackage{amsmath}


\setcounter{MaxMatrixCols}{10}
%TCIDATA{OutputFilter=LATEX.DLL}
%TCIDATA{Version=5.00.0.2552}
%TCIDATA{<META NAME="SaveForMode" CONTENT="1">}
%TCIDATA{Created=Tuesday, October 13, 2015 14:51:37}
%TCIDATA{LastRevised=Thursday, November 26, 2015 12:43:21}
%TCIDATA{<META NAME="GraphicsSave" CONTENT="32">}
%TCIDATA{<META NAME="DocumentShell" CONTENT="Standard LaTeX\Blank - Standard LaTeX Article">}
%TCIDATA{CSTFile=40 LaTeX article.cst}
%TCIDATA{ComputeDefs=
%$F=kT\frac{1-2\frac{N}{{}}}{{}}$
%}


\newtheorem{theorem}{Theorem}
\newtheorem{acknowledgement}[theorem]{Acknowledgement}
\newtheorem{algorithm}[theorem]{Algorithm}
\newtheorem{axiom}[theorem]{Axiom}
\newtheorem{case}[theorem]{Case}
\newtheorem{claim}[theorem]{Claim}
\newtheorem{conclusion}[theorem]{Conclusion}
\newtheorem{condition}[theorem]{Condition}
\newtheorem{conjecture}[theorem]{Conjecture}
\newtheorem{corollary}[theorem]{Corollary}
\newtheorem{criterion}[theorem]{Criterion}
\newtheorem{definition}[theorem]{Definition}
\newtheorem{example}[theorem]{Example}
\newtheorem{exercise}[theorem]{Exercise}
\newtheorem{lemma}[theorem]{Lemma}
\newtheorem{notation}[theorem]{Notation}
\newtheorem{problem}[theorem]{Problem}
\newtheorem{proposition}[theorem]{Proposition}
\newtheorem{remark}[theorem]{Remark}
\newtheorem{solution}[theorem]{Solution}
\newtheorem{summary}[theorem]{Summary}
\newenvironment{proof}[1][Proof]{\noindent\textbf{#1.} }{\ \rule{0.5em}{0.5em}}


\begin{document}



\bigskip 统力第五周第一次%
作业\qquad \qquad \qquad 赵丰\qquad \qquad 2013012178


\bigskip 设体系只对外做体%
积功PdV,H=E+PV=Const,对此状态函%
数取微分有dE+PdV+VdP=0,

由Clausius 熵公式,孤立系可%
逆过程dS=$\frac{dQ}{T}$,由First Law of
Thermodynamics 有dQ=dE+PdV$\implies $

dS=$\frac{dE+PdV}{T}=\frac{-VdP}{T}\implies \left( \frac{\partial S}{%
\partial P}\right) _{H}=\frac{-V}{T}<0.$

若E=const,则dS=$\frac{PdV}{T}\implies \left( \frac{\partial S}{%
\partial V}\right) _{E}=\frac{P}{T}>0.$


$\left( 2\right) $解:使用微正则系%
综的方法$\left( N,L\left( \approx V\right) ,E\text{%
恒定}\right) $,先求满足条件%
的微观状态数,设各链%
环两种取向分别为+和-,%
有$n_{1}$个链环取正向,$n_{2}$%
个链环取负向,则$%
n_{1}+n_{2}=N,a\left( n_{1}-n_{2}\right) =L$,从而解出%
$n_{1}=\frac{N+\frac{l}{a}}{2},n_{2}=\frac{N\acute{-}\frac{l}{a}}{2}.$%
设$x=\frac{n_{1}}{N}=\frac{N+\frac{l}{a}}{2N}.$由Boltzmann%
熵的定义$S=k\ln \Omega $,而$\Omega =%
\binom{N}{n_{1}}=\frac{N!}{n_{1}!\left( N-n_{1}\right) !},$by Stirling's
formula, we can approximate ln$\Omega $ as follows:

$\ln \Omega =\ln N!-\ln n_{1}!-\ln \left( N-n_{1}\right) !=N\left( \ln
N-1\right) -n_{1}\left( \ln n_{1}-1\right) -n_{2}\left( \ln n_{2}-1\right)
=N\ln N-n_{1}\ln n_{1}-n_{2}\ln n_{2}$

$N\left[ x\ln N+\left( 1-x\right) \ln N-x\ln \left( Nx\right) -\left(
1-x\right) \ln \left( \left( 1-x\right) N\right) \right] =-N\left[ x\ln
x+\left( 1-x\right) \ln \left( 1-x\right) \right] .$

By the definition of temperature we have:

$\frac{1}{T}=\left( \frac{\partial S}{\partial E}\right) _{N,L}=k\left( 
\frac{\partial \ln \Omega }{\partial E}\right) _{N,L}=k\left( \frac{\partial
\ln \Omega }{\partial x}\right) _{N,L}\left( \frac{\partial x}{\partial L}%
\right) _{N,L}\left( \frac{\partial L}{\partial E}\right) _{N,L}$

=$k\left( N\ln \frac{1-x}{x}\right) \frac{1}{2Na}\left( \frac{\partial L}{%
\partial E}\right) _{N,L}\implies $

the general force $F=\left( \frac{\partial E}{\partial L}\right)
_{N,L}=kT\left( \ln \frac{1-x}{x}\right) \frac{1}{2a}.$

As $L<<N_{A},x$ is very near to $\frac{1}{2}$ as shown by $x=\frac{N+\frac{l%
}{a}}{2N}.$

$\ln \frac{1-x}{x}=\ln \left( 1+\frac{1-2x}{x}\right) \approx \frac{1-2x}{x}%
, $by Taylor approximation since $\frac{1-2x}{x}\approx 0.$

Then $F=kTN\frac{1-2\frac{N+\frac{l}{a}}{2N}}{\frac{N+\frac{l}{a}}{2N}}\frac{%
1}{2Na}=\allowbreak -Tk\frac{l}{al+Na^{2}}\approx -\frac{Tk}{Na^{2}}l$
satisfying Hook's Law$,$the negative sign shows that F is negative to the
x-axis when l increases (notice that F is the force acted on the system by
surrounding.)

We can also use the canonical ensemble (N,L,T of the system are fixed) to
calculate F. First we calculate the partition function.

Since the total energy is fixed by the fixed N and the single energy level
of each "particle".But we have many states of degeneracy. As show in the
solution by micro-canonical ensemble, the degeneracy equals eactly $\binom{N%
}{n_{1}},$that is $\Omega $ calculated above$\implies Z=\Omega e^{-\beta E}.$%
By the thermodynamic formula for canonical ensemble

$F=-\frac{1}{\beta }\frac{\partial \ln Z}{\partial y}.$Notice that we choose 
$L$ as the generalized coordinate, which decreases along the action path of
the exterior force, similar to volume as generalized coordinate. $\implies $

$F=kT\frac{\partial \ln \Omega }{\partial L},$which produces the same
results as above.


解\qquad $\left( 1\right) $对于N个正常%
位置有n个有晶体缺失%
共有$\binom{N}{n}$种不同的状%
态数,对于填隙同理有$%
\binom{N}{n}$种不同的状态数

$S=k\ln \Omega =k\ln \binom{N}{n}^{2}=2k\ln \frac{N!}{n!\left( N-n\right) !}%
. $

$\left( 2\right) F=nu-TS=nu-2Tk\ln \frac{N!}{n!\left( N-n\right) !},$by
Stirling's formula we have

$\ln \frac{N!}{n!\left( N-n\right) !}\approx N\ln N-n\ln n-\left( N-n\right)
\ln \left( N-n\right) \qquad \left( \text{similar to the above problem}%
\right) $.

Now assume F is continuous depedent on n and taking the derative of F about
n gives

$\frac{dF}{dn}=u-2Tk\ln \left( \frac{N}{n}-1\right) .$ F has minimum at some
n$\implies u-2Tk\ln \left( \frac{N}{n}-1\right) =0$

$\implies \frac{N}{n}=e^{u/\left( 2Tk\right) }+1.$

$N>>n\implies u/\left( 2Tk\right) >>1$

Hence we can approximate $\frac{N}{n}$ by $e^{u/\left( 2Tk\right) }\implies $

n$\approx Ne^{-\frac{u}{2kT}}.$


$\left( 1\right) $表面膜的内能变$%
dE=TdS+\sigma dA\left( TdS\text{相当于\dj }Q,\text{表%
面积变也引起}dE\right) ,$由$E$ is
extensive$\left( E\left( \lambda S,\lambda A\right) =\lambda E\left(
S,A\right) \right) $ and

Euler's Thm$\implies E=ST+\sigma A.\implies SdT+Ad\sigma =0\implies
S=-A\left( \frac{d\sigma }{dT}\right) $自由能为$%
F=E-TS=\sigma A.$

$\left( 2\right) $\dj $Q=TdS=-T\left( \frac{d\sigma }{dT}\right) dA\implies
Q=-T\left( \frac{d\sigma }{dT}\right) \Delta A($dA 相当于%
表面膜原子数从bulk中增%
加$,$d$\sigma $是在表面膜

原子数不变strain的变化)

$\left( 3\right) E=\sigma \left( T\right) A-TA\left( \frac{d\sigma }{dT}%
\right) ,$对可逆绝热过程,$%
dE=\sigma dA,$认为$A$与$T$为独立变%
量

$\implies \sigma ^{\prime }\left( T\right) AdT-A\left( \frac{d\sigma }{dT}%
\right) dT-TA\sigma ^{\prime \prime }\left( T\right) dT+\sigma \left(
T\right) dA-T\left( \frac{d\sigma }{dT}\right) dA=\sigma dA$

\bigskip $\implies A\sigma ^{\prime \prime }\left( T\right) dT+\left( \frac{%
d\sigma }{dT}\right) dA=0\implies \frac{\sigma ^{\prime \prime }\left(
T\right) dT}{\sigma ^{\prime }\left( T\right) }+\frac{dA}{A}=0\implies
\sigma \left( T\right) =\frac{C}{A},$

where $C$ is an constant$.$

磁滞损耗$\int \mu _{0}HdM$即为磁%
化时放出的热量,考虑%
到$M=C\frac{H}{T},T=const.$其大小为$Q=\int
C\mu _{0}H\frac{dH}{T}=\frac{\mu _{0}CH^{2}}{2T}.$

设每一个粒子具有完%
全相同的能量E,将小体%
积v视为考虑的系统,由%
巨正则分布的概率分%
布可知v处于有确定的n%
个粒子的态时几率为$\Xi
^{-1}e^{-\alpha n-\beta n}=\Xi ^{-1}e^{-cn}\left( \text{乘以n%
的全排列是因为总能%
量为}nE\right) .$而P$_{n}$表示任意$%
n$个粒子在v中的概率$,$%
因此

$P_{n}=\binom{N}{n}\Xi ^{-1}e^{-cn},$因为$N>>n,$考虑%
到$P_{n}=\binom{N}{n}\Xi ^{-1}e^{-cn}=\frac{N!}{n!\left( N-n\right) !}%
\Xi ^{-1}e^{-cn}=\frac{\left( 1-\frac{1}{N}\right) ..\left( 1-\frac{n-1}{N}%
\right) N^{n}}{n!}\Xi ^{-1}e^{-cn}$

$\approx \frac{N^{n}}{n!}\Xi ^{-1}e^{-cn}=\frac{1}{n!}\Xi ^{-1}\lambda ^{n}.$%
由归一化条件得$\Xi =e^{\lambda
}\implies P_{n}=\frac{\lambda ^{n}}{n!}e^{-\lambda }$对P$_{n}$求%
均值得$E\left( P_{n}\right) =\underset{n=0}{\overset{%
\infty }{\sum }}nP_{n}=\lambda ,$即$\lambda =\overline{n}.$所%
以有$P_{n}=\frac{\overline{n}^{n}}{n!}e^{-\overline{n}}$

此题也可用Binomial Distribution 来%
近似,假设每个粒子在V%
中服从Uniform Distribution,且不同粒%
子分布相互独立,则对%
任一粒子,其落入v体积%
此随机变量服从Bernoulli
Distribution,"成功"的概率不妨%
记为p(=$\frac{v}{V}).$则落入v中的%
粒子数此随机变量X服%
从B(N,p),N为总粒子数,相当%
于实验次数.故有P$\left( X=n\right)
=\binom{N}{n}p^{n}\left( 1-p\right) ^{N-n}$,由于N\TEXTsymbol{>}%
\TEXTsymbol{>}n,而Np可近似看成常%
数$\lambda \left( \underset{N->\infty }{\lim }Np=\lambda ,\text{p}=%
\frac{v}{V}\right) $,由于Binomial Distribution的极%
限情形(取热力学极限N-%
\TEXTsymbol{>}$\infty ,$此时V-\TEXTsymbol{>}$\infty )$即%
为Poission Distribution$\implies P_{n}=\underset{N->\infty }{\lim }%
P\left( X=n\right) =\frac{\lambda ^{n}}{n!}e^{-\lambda }.$


\end{document}
