
\documentclass{ctexart}
\usepackage{amsmath}


%%%%%%%%%%%%%%%%%%%%%%%%%%%%%%%%%%%%%%%%%%%%%%%%%%%%%%%%%%%%%%%%%%%%%%%%%%%%%%%%%%%%%%%%%%%%%%%%%%%%%%%%%%%%%%%%%%%%%%%%%%%%%%%%%%%%%%%%%%%%%%%%%%%%%%%%%%%%%%%%%%%%%%%%%%%%%%%%%%%%%%%%%%%%%%%%%%%%%%%%%%%%%%%%%%%%%%%%%%%%%%%%%%%%
%TCIDATA{OutputFilter=LATEX.DLL}
%TCIDATA{Version=5.00.0.2552}
%TCIDATA{<META NAME="SaveForMode" CONTENT="1">}
%TCIDATA{Created=Tuesday, November 10, 2015 02:52:24}
%TCIDATA{LastRevised=Tuesday, November 10, 2015 23:17:10}
%TCIDATA{<META NAME="GraphicsSave" CONTENT="32">}
%TCIDATA{<META NAME="DocumentShell" CONTENT="Standard LaTeX\Blank - Standard LaTeX Article">}
%TCIDATA{CSTFile=40 LaTeX article.cst}
%TCIDATA{ComputeDefs=
%$S=k\left( \alpha \bar{N}+\beta \bar{E}+\ln \Xi \right) =k\left( \bar{N}\ln 
%\frac{z}{\bar{N}}+\frac{3}{2}\bar{N}+\bar{N}\right) =k\left( \bar{N}\ln
%\right) $
%}


\newtheorem{theorem}{Theorem}
\newtheorem{acknowledgement}[theorem]{Acknowledgement}
\newtheorem{algorithm}[theorem]{Algorithm}
\newtheorem{axiom}[theorem]{Axiom}
\newtheorem{case}[theorem]{Case}
\newtheorem{claim}[theorem]{Claim}
\newtheorem{conclusion}[theorem]{Conclusion}
\newtheorem{condition}[theorem]{Condition}
\newtheorem{conjecture}[theorem]{Conjecture}
\newtheorem{corollary}[theorem]{Corollary}
\newtheorem{criterion}[theorem]{Criterion}
\newtheorem{definition}[theorem]{Definition}
\newtheorem{example}[theorem]{Example}
\newtheorem{exercise}[theorem]{Exercise}
\newtheorem{lemma}[theorem]{Lemma}
\newtheorem{notation}[theorem]{Notation}
\newtheorem{problem}[theorem]{Problem}
\newtheorem{proposition}[theorem]{Proposition}
\newtheorem{remark}[theorem]{Remark}
\newtheorem{solution}[theorem]{Solution}
\newtheorem{summary}[theorem]{Summary}
\newenvironment{proof}[1][Proof]{\noindent\textbf{#1.} }{\ \rule{0.5em}{0.5em}}


\begin{document}



正则系综导出单原子%
理想气体$E$和$S$的表达%
式

单粒子配分函数为$z=V\left( 
\frac{2\pi m}{\beta h^{2}}\right) ^{3/2},$在双原子%
的情形,此式对应为平%
动配分函数$z_{t}.$

$E=-N\frac{\partial \ln z}{\partial \beta }=\frac{3}{2}NkT.$

$S=k\left( \ln Z+\beta E\right) ,$因粒子不可%
分辨 $Z=\frac{z^{N}}{N!},$

$S=k\left( N\ln z-\left( \ln N-1\right) N+\frac{3}{2}N\right) $

=$kN\ln \frac{z}{N}+\frac{5}{2}N=kN\ln \frac{V}{N}\left( \frac{2\pi m}{\beta
h^{2}}\right) ^{3/2}+\frac{5}{2}kN.$

对双原子分子,在常温%
下振动配分函数对$z_{v}$%
贡献很小,只考虑$z_{t}$和$%
z_{r}=\frac{8\pi ^{2}I}{\beta h^{2}}.z=z_{t}z_{r}$

$E=-N\frac{\partial \ln z}{\partial \beta }=\frac{5}{2}NkT\implies C_{v}=%
\frac{5}{2}kT.$

$S=k\left( \ln Z+\beta E\right) ,$因粒子不可%
分辨 $Z=\frac{z_{t}^{N}z_{r}^{N}}{N!},$

$S=k\left( N\ln z-\left( \ln N-1\right) N+\frac{5}{2}N\right) $

=$kN\ln \frac{z}{N}+\frac{7}{2}N=kN\ln \frac{V}{N}\frac{C}{\beta ^{5/2}}+%
\frac{5}{2}kN.$

巨正则系综导出单原%
子理想气体$E$和$S$的表%
达式

$\Xi =\underset{N=0}{\overset{\infty }{\sum }}e^{-\alpha N-\beta E},$准%
经典近似下

$\Xi =\underset{N=0}{\overset{\infty }{\sum }}\frac{1}{N!h^{3N}}\int
e^{-\alpha N-\beta E}d\omega =\underset{N=0}{\overset{\infty }{\sum }}\frac{%
e^{-\alpha N}z^{N}}{N!}=\exp \left( e^{-\alpha }z\right) $

$\implies \ln \Xi =e^{-\alpha }z.$

平均粒子数$\bar{N}=-\frac{\partial \ln \Xi }{%
\partial \alpha }=e^{-\alpha }z=\ln \Xi .$

$\bar{E}=-\frac{\partial \ln \Xi }{\partial \beta }=\frac{3}{2\beta }%
e^{-\alpha }z=\frac{3}{2}\bar{N}kT.$

$S=k\left( \alpha \bar{N}+\beta \bar{E}+\ln \Xi \right) =k\left( \bar{N}\ln 
\frac{z}{\bar{N}}+\frac{3}{2}\bar{N}+\bar{N}\right) =k\bar{N}\ln \frac{V}{N}%
\left( \frac{2\pi m}{\beta h^{2}}\right) ^{3/2}+\frac{5}{2}k\bar{N}$

用巨正则系综求熵最%
简便.

三维相对论气体$\left( \epsilon
=cp\right) $配分函数的计算,设%
气体体积为V,粒子数为N,%
粒子可分辨:

$z=\frac{1}{h^{3}}\int e^{-\beta \epsilon }d\vec{\omega}=\frac{V}{h^{3}}%
\int_{0}^{\infty }e^{-\beta cp}4\pi p^{2}dp=\frac{8\pi V}{h^{3}c^{3}\beta
^{3}}$

$\implies E=3NkT,$相对论气体仍满%
足物态方程$pV=NkT.$


\end{document}
