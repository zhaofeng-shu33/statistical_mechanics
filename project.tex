
\documentclass{ctexart}
\usepackage{amsmath}



\newtheorem{theorem}{Theorem}
\newtheorem{acknowledgement}[theorem]{Acknowledgement}
\newtheorem{algorithm}[theorem]{Algorithm}
\newtheorem{axiom}[theorem]{Axiom}
\newtheorem{case}[theorem]{Case}
\newtheorem{claim}[theorem]{Claim}
\newtheorem{conclusion}[theorem]{Conclusion}
\newtheorem{condition}[theorem]{Condition}
\newtheorem{conjecture}[theorem]{Conjecture}
\newtheorem{corollary}[theorem]{Corollary}
\newtheorem{criterion}[theorem]{Criterion}
\newtheorem{definition}[theorem]{Definition}
\newtheorem{example}[theorem]{Example}
\newtheorem{exercise}[theorem]{Exercise}
\newtheorem{lemma}[theorem]{Lemma}
\newtheorem{notation}[theorem]{Notation}
\newtheorem{problem}[theorem]{Problem}
\newtheorem{proposition}[theorem]{Proposition}
\newtheorem{remark}[theorem]{Remark}
\newtheorem{solution}[theorem]{Solution}
\newtheorem{summary}[theorem]{Summary}
\newenvironment{proof}[1][Proof]{\noindent\textbf{#1.} }{\ \rule{0.5em}{0.5em}}
\input{tcilatex}

\begin{document}
\begin{CJK}{GBK}{song}
N维球体体积公式简便%
推导:

设$\Sigma y=\underset{\underset{i=1}{\overset{n}{\Sigma }}%
x_{i}^{2}\leq y}{\didotsint }dx_{1}..dx_{n},$

变量替换$\implies \Sigma y=Ky^{n/2},$where $K=%
\underset{\underset{i=1}{\overset{n}{\Sigma }}x_{i}^{2}\leq 1}{\didotsint }%
dx_{1}..dx_{n}$ is a constant determined by n.

Let $I=\underset{\NEG{R}^{n}}{\didotsint }e^{-\underset{i=1}{\overset{n}{%
\Sigma }}x_{i}^{2}}dx_{1}..dx_{n},$ we give two methods to calculate $I.$

Firstly we can use Fubini's Thm and integrate $I$ by parts:

$I=\left( \int_{-\infty }^{\infty }e^{-x^{2}}dx\right) ^{n}=\pi ^{n/2}.$

Secondly we can make substitutions: $y=\underset{i=1}{\overset{n}{\Sigma }}%
x_{i}^{2}.$

Notice that we can understand $I$ as a limit of summation:

$I=\sum \underset{y<\underset{i=1}{\overset{n}{\Sigma }}x_{i}^{2}<y+\Delta y}%
{\didotsint }e^{-\underset{i=1}{\overset{n}{\Sigma }}%
x_{i}^{2}}dx_{1}..dx_{n}=\sum \underset{y<\underset{i=1}{\overset{n}{\Sigma }%
}x_{i}^{2}<y+\Delta y}{\didotsint }e^{-y}dx_{1}..dx_{n}=\sum e^{-y}\underset{%
y<\underset{i=1}{\overset{n}{\Sigma }}x_{i}^{2}<y+\Delta y}{\didotsint }%
dx_{1}..dx_{n}$

=$\sum e^{-y}d\Sigma y,$where $d\Sigma y=\underset{\underset{i=1}{\overset{n}%
{\Sigma }}x_{i}^{2}<y+\Delta y}{\didotsint }dx_{1}..dx_{n}-\underset{%
\underset{i=1}{\overset{n}{\Sigma }}x_{i}^{2}<y}{\didotsint }dx_{1}..dx_{n}.$

As $\Delta y$ is very small, the summation can be replaced by integral sign,
we have

$I=\int_{0}^{\infty }e^{-y}d\Sigma y.$ We can calculate the differential
from $\Sigma y=Ky^{n/2}\implies $

$I=K\frac{n}{2}\int_{0}^{\infty }y^{\frac{n}{2}-1}e^{-y}dy=K\frac{n}{2}%
\Gamma \left( \frac{n}{2}\right) =K\Gamma \left( \frac{n}{2}+1\right) .$

Comparing the results from the two methods gives $K=\frac{\pi ^{n/2}}{\Gamma
\left( \frac{n}{2}+1\right) }.$

此法可照搬用于推导n%
维单形$\underset{\underset{i=1}{\overset{n}{\Sigma }}%
x_{i}\leq 1\&x_{i}\geq 0}{\didotsint }dx_{1}..dx_{n}=\frac{1}{n!}.$

\bigskip

第四周作业题另一种%
可能的情况$\left( \text{粒子%
不可分辨}\right) $:

其中$E_{i}$为系统处于某%
一微观态时所对应的%
能量值,此题中$E_{i}$由$N$个%
粒子占据的能级确定$,$%
可由下式计算

$Z=\underset{i+j\leq N}{\sum }e^{\beta \left( i\epsilon _{1}+j\epsilon
_{2}\right) }=\underset{i=0}{\overset{N}{\sum }}\underset{j=0}{\overset{N-i}{%
\sum }}e^{\beta \left( i\epsilon _{1}+j\epsilon _{2}\right) }=\underset{i=0}{%
\overset{N}{\sum }}e^{\beta i\epsilon _{1}}\underset{j=0}{\overset{N-i}{\sum 
}}e^{\beta j\epsilon _{2}}$

=$\underset{i=0}{\overset{N}{\sum }}e^{\beta i\epsilon _{1}}\frac{1-e^{\beta
\epsilon _{2}(N+1-i)}}{1-e^{\beta \epsilon _{2}}}=\frac{1}{1-e^{\beta
\epsilon _{2}}}(\underset{i=0}{\overset{N}{\sum }}e^{\beta i\epsilon
_{1}}-e^{\beta \epsilon _{2}(N+1)}\underset{i=0}{\overset{N}{\sum }}e^{\beta
i\left( \epsilon _{1}-\epsilon _{2}\right) })$

=$\frac{1}{1-e^{\beta \epsilon _{2}}}(\frac{1-e^{\beta \epsilon _{1}\left(
N+1\right) }}{1-e^{\beta \epsilon _{1}}}-e^{\beta \epsilon _{2}(N+1)}\frac{%
1-e^{\beta \left( \epsilon _{1}-\epsilon _{2}\right) \left( N+1\right) }}{%
1-e^{\beta \left( \epsilon _{1}-\epsilon _{2}\right) }})=$
\end {CJK}
\end{document}
