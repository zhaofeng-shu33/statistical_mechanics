
\documentclass{ctexart}
\usepackage{amsmath}


%%%%%%%%%%%%%%%%%%%%%%%%%%%%%%%%%%%%%%%%%%%%%%%%%%%%%%%%%%%%%%%%%%%%%%%%%%%%%%%%%%%%%%%%%%%%%%%%%%%%%%%%%%%%%%%%%%%%%%%%%%%%%%%%%%%%%%%%%%%%%%%%%%%%%%%%%%%%%%%%%%%%%%%%%%%%%%%%%%%%%%%%%%%%%%%%%%%%%%%%%%%%%%%%%%%%%%%%%%%%%%%%%%%%
%TCIDATA{OutputFilter=LATEX.DLL}
%TCIDATA{Version=5.00.0.2552}
%TCIDATA{<META NAME="SaveForMode" CONTENT="1">}
%TCIDATA{Created=Friday, November 06, 2015 08:16:59}
%TCIDATA{LastRevised=Friday, November 06, 2015 12:16:24}
%TCIDATA{<META NAME="GraphicsSave" CONTENT="32">}
%TCIDATA{<META NAME="DocumentShell" CONTENT="Standard LaTeX\Blank - Standard LaTeX Article">}
%TCIDATA{CSTFile=40 LaTeX article.cst}

\newtheorem{theorem}{Theorem}
\newtheorem{acknowledgement}[theorem]{Acknowledgement}
\newtheorem{algorithm}[theorem]{Algorithm}
\newtheorem{axiom}[theorem]{Axiom}
\newtheorem{case}[theorem]{Case}
\newtheorem{claim}[theorem]{Claim}
\newtheorem{conclusion}[theorem]{Conclusion}
\newtheorem{condition}[theorem]{Condition}
\newtheorem{conjecture}[theorem]{Conjecture}
\newtheorem{corollary}[theorem]{Corollary}
\newtheorem{criterion}[theorem]{Criterion}
\newtheorem{definition}[theorem]{Definition}
\newtheorem{example}[theorem]{Example}
\newtheorem{exercise}[theorem]{Exercise}
\newtheorem{lemma}[theorem]{Lemma}
\newtheorem{notation}[theorem]{Notation}
\newtheorem{problem}[theorem]{Problem}
\newtheorem{proposition}[theorem]{Proposition}
\newtheorem{remark}[theorem]{Remark}
\newtheorem{solution}[theorem]{Solution}
\newtheorem{summary}[theorem]{Summary}
\newenvironment{proof}[1][Proof]{\noindent\textbf{#1.} }{\ \rule{0.5em}{0.5em}}


\begin{document}



$\left( 3\right) $求最可几分布的%
几率$p\left( \frac{N}{2}\right) ,$假设n是%
最可几分布$\frac{N}{2}$的14亿%
分之一即$\left( n\approx 10^{13}\right) ,$

问分布的几率p$\left( \frac{N}{2}\pm
n\right) $是多少?

由此说明,最可几分布%
代表系统的平衡分布.

$p\left( \frac{N}{2}\right) =\frac{N!}{\left( \frac{N}{2}!\right) ^{2}}%
\left( \frac{1}{2}\right) ^{N},N\rightarrow \infty .$

$\ln p\left( \frac{N}{2}\right) \rightarrow 0\implies p\left( \frac{N}{2}%
\right) \rightarrow 1.$

p$\left( \frac{N}{2}\pm n\right) \approx e^{-14^{2}\times 10^{8}}.$

二维理想气体,单个分%
子的能量在d$\epsilon $范围内%
的几率是$\rho \left( \epsilon \right) =ce^{-\beta
\epsilon }d\epsilon ,$c is a constant.

试求:任选两个分子,其%
总能量在dE范围内的几%
率,并证\={E}=2kT.

\bigskip 

在面积为S的薄膜上浮%
有N个有机分子,分子整%
体可视作是二维理想%
气体,气体的温度为T,试%
求气体的定面积热容%
量$\left( \text{整个有机分子%
可视为单原子分子}\right) $

先求体系的partition function.$z=\frac{1}{h^{2}}%
\left( \frac{2\pi m}{\beta }\right) \int e^{-\beta c}dxdy\implies N=NkT.$

\bigskip 

若单原子分子的能量$%
\epsilon =ap^{2},-$其中a是常数,n$\in N,$p%
是分子的动量,证明气%
体的能量\={E}=$\frac{nE}{3V}.$

z=$\frac{V}{h^{2}}\int e^{-\beta \epsilon }d\vec{p}=\frac{4\pi V}{h^{2}}%
\int_{0}^{\infty }e^{-a\beta p^{n}}p^{2}dp=\frac{4\pi V}{h^{2}}%
\int_{0}^{\infty }e^{-a\beta p^{n}}\frac{p^{2}}{n\beta ap^{n-1}}d\left(
a\beta p^{n}\right) =\frac{4\pi V}{h^{2}}\int_{0}^{\infty }e^{-t}\frac{{}}{%
n\beta a\left( \frac{t}{a\beta }\right) ^{\frac{n-3}{n}}}dt$

=$\frac{4\pi V}{h^{2}}\int_{0}^{\infty }e^{-t}\frac{\left( a\beta \right)
^{\left( n-3\right) /n}}{n\beta a}t^{\frac{-\left( n-3\right) }{n}}dt=\frac{%
4\pi V}{h^{2}}\frac{\left( a\beta \right) ^{\left( n-3\right) /n}}{n\beta a}%
\int_{0}^{\infty }e^{-t}t^{\frac{-\left( n-3\right) }{n}}dt=\frac{4\pi V}{%
h^{2}}\frac{\left( a\beta \right) ^{\left( n-3\right) /n}}{n\beta a}%
\int_{0}^{\infty }e^{-t}t^{\frac{-\left( n-3\right) }{n}}dt$

$=\Gamma \left( \frac{3}{n}\right) \frac{4\pi V}{h^{2}}\frac{a^{-3/n}}{n}%
\beta ^{-3/n}\implies \bar{E}=-N\frac{\partial \ln z}{\partial \beta }=\frac{%
3N}{n\beta },\bar{p}=\frac{N}{\beta }\frac{\partial \ln z}{\partial V}=\frac{%
N}{\beta V}=\frac{N}{V}\frac{n\bar{E}}{3N}=\frac{n\bar{E}}{3V}.$

相对论n=1,非相对论n=2.

由N个近独立的三维谐%
振子组成的系统,振子%
能级$\epsilon _{n}=\left( n+\frac{3}{2}\right) h\nu ,$简%
并度$\epsilon _{n}=\frac{1}{2}\left( n+1\right) \left(
n+2\right) $推Bose分布时有类似,1%
证:三维谐振子的partition function
is 三次方 of the partition function of one-dimensional
harmonic oscillator.

$z_{3}=\sum z_{n}e^{-\beta \left( n+\frac{3}{2}\right) h\nu }=e^{-3\beta
h\nu /2}\sum \frac{1}{2}\left( n+1\right) \left( n+2\right) x^{n},$where $%
x=e^{-\beta h\nu }.$

=$e^{-3\beta h\nu /2}\sum \frac{1}{2}\frac{d^{2}}{dx^{2}}x^{n+2}=\frac{1}{2}%
e^{-3\beta h\nu /2}\frac{d^{2}}{dx^{2}}\overset{\infty }{\underset{n=0}{\sum 
}}x^{n+2}=\frac{1}{2}e^{-3\beta h\nu /2}\frac{d^{2}}{dx^{2}}\frac{x^{2}}{1-x}%
=$-(2/(-1 + x)\symbol{94}3)

\bigskip 

试求在绝对零度下,二%
维Fermi气体的Fermi energy $\epsilon _{F}$和%
单个粒子的平均能量$%
\bar{\epsilon}_{0}\left( \text{设粒子的总%
数为N}\right) $


\end{document}
