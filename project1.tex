
\documentclass{ctexart}
%%%%%%%%%%%%%%%%%%%%%%%%%%%%%%%%%%%%%%%%%%%%%%%%%%%%%%%%%%%%%%%%%%%%%%%%%%%%%%%%%%%%%%%%%%%%%%%%%%%%%%%%%%%%%%%%%%%%%%%%%%%%%%%%%%%%%%%%%%%%%%%%%%%%%%%%%%%%%%%%%%%%%%%%%%%%%%%%%%%%%%%%%%%%%%%%%%%%%%%%%%%%%%%%%%%%%%%%%%%%%%%%%%%%%%%%%%%%%%%%%%%%%%%%%%%%
\usepackage{amsmath}

\setcounter{MaxMatrixCols}{10}
%TCIDATA{OutputFilter=LATEX.DLL}
%TCIDATA{Version=5.00.0.2552}
%TCIDATA{<META NAME="SaveForMode" CONTENT="1">}
%TCIDATA{Created=Thursday, October 15, 2015 23:40:56}
%TCIDATA{LastRevised=Sunday, October 18, 2015 20:56:23}
%TCIDATA{<META NAME="GraphicsSave" CONTENT="32">}
%TCIDATA{<META NAME="DocumentShell" CONTENT="Scientific Notebook\Blank Document">}
%TCIDATA{CSTFile=Math with theorems suppressed.cst}
%TCIDATA{PageSetup=72,72,72,72,0}
%TCIDATA{AllPages=
%F=36,\PARA{038<p type="texpara" tag="Body Text" >\hfill \thepage}
%}


\newtheorem{theorem}{Theorem}
\newtheorem{acknowledgement}[theorem]{Acknowledgement}
\newtheorem{algorithm}[theorem]{Algorithm}
\newtheorem{axiom}[theorem]{Axiom}
\newtheorem{case}[theorem]{Case}
\newtheorem{claim}[theorem]{Claim}
\newtheorem{conclusion}[theorem]{Conclusion}
\newtheorem{condition}[theorem]{Condition}
\newtheorem{conjecture}[theorem]{Conjecture}
\newtheorem{corollary}[theorem]{Corollary}
\newtheorem{criterion}[theorem]{Criterion}
\newtheorem{definition}[theorem]{Definition}
\newtheorem{example}[theorem]{Example}
\newtheorem{exercise}[theorem]{Exercise}
\newtheorem{lemma}[theorem]{Lemma}
\newtheorem{notation}[theorem]{Notation}
\newtheorem{problem}[theorem]{Problem}
\newtheorem{proposition}[theorem]{Proposition}
\newtheorem{remark}[theorem]{Remark}
\newtheorem{solution}[theorem]{Solution}
\newtheorem{summary}[theorem]{Summary}
\newenvironment{proof}[1][Proof]{\noindent\textbf{#1.} }{\ \rule{0.5em}{0.5em}}


\begin{document}


\bigskip 统力第四周作业 \qquad 
赵丰\qquad 2013012178

\bigskip \bigskip 统力第五周第二%
次作业 \qquad 赵丰\qquad 2013012178

\bigskip 



因$\epsilon _{2}>>\epsilon _{1}$,故体系不%
存在能级的简并,partition function 
由下式给出

$Z=\underset{i}{\sum }e^{\beta E_{i}},$若我们假%
设粒子是可分辨的,则N%
个粒子的微观态分布%
为独立同分布,其联合%
微观态的分布等于单%
一分布的N次方,即有\qquad $%
Z=\left( 1+e^{-\beta \epsilon _{1}}+e^{-\beta \epsilon _{2}}\right) ^{N}.$

Helmholtz自由能为\qquad $F=-kT\ln Z=-kTN\ln \left(
1+e^{-\beta \epsilon _{1}}+e^{-\beta \epsilon _{2}}\right) $

而系统的熵为\qquad $S=k\left[ \ln
Z-\beta \frac{\partial \ln Z}{\partial \beta }\right] =Nk\left[ \ln \left(
1+e^{-\beta \epsilon _{1}}+e^{-\beta \epsilon _{2}}\right) +\frac{\beta
\epsilon _{1}e^{-\beta \epsilon _{1}}+\beta \epsilon _{2}e^{-\beta \epsilon
_{2}}}{1+e^{-\beta \epsilon _{1}}+e^{-\beta \epsilon _{2}}}\right] $

$\left( 1\right) kT<<\epsilon _{2},$即$\beta \epsilon _{2}>>1,F\approx
-kTN\ln \left( 1+e^{-\beta \epsilon _{1}}\right) $

$S\approx Nk\left[ \ln \left( 1+e^{-\beta \epsilon _{1}}\right) +\frac{\beta
\epsilon _{1}e^{-\beta \epsilon _{1}}}{1+e^{-\beta \epsilon _{1}}}\right] $

\bigskip $\left( 2\right) kT>>\epsilon _{1},即 \beta \epsilon
_{1}<<1,F\approx -kTN\ln \left( 2-\beta \epsilon _{1}+e^{-\beta \epsilon
_{2}}\right) $

$S\approx Nk\left[ \ln \left( 2-\beta \epsilon _{1}+e^{-\beta \epsilon
_{2}}\right) +\frac{\beta \epsilon _{1}+\beta \epsilon _{2}e^{-\beta
\epsilon _{2}}}{2-\beta \epsilon _{1}+e^{-\beta \epsilon _{2}}}\right] $

\bigskip

\bigskip 若用grand canonical ensemble$\left( T,\mu
,V=Const\right) :$

巨热力学势$\Xi =\underset{N=0}{\overset{\infty 
}{\sum }}e^{-\alpha N}Z\left( N\right) =\underset{n=0}{\overset{\infty }{%
\sum }}e^{-\alpha N}\left( 1+e^{-\beta \epsilon _{1}}+e^{-\beta \epsilon
_{2}}\right) ^{N}=\frac{1}{1-e^{-\alpha }\left( 1+e^{-\beta \epsilon
_{1}}+e^{-\beta \epsilon _{2}}\right) }.$

平均粒子数$\overline{N}=-\left( \frac{\partial
\ln \Xi }{\partial \alpha }\right) _{T,V}=\frac{\partial \ln \left(
1-e^{-\alpha }\left( 1+e^{-\beta \epsilon _{1}}+e^{-\beta \epsilon
_{2}}\right) \right) }{\partial \alpha }=\allowbreak \frac{e^{-\alpha
}\left( 1+e^{-\beta \epsilon _{1}}+e^{-\beta \epsilon _{2}}\right) }{%
1-e^{-\alpha }\left( 1+e^{-\beta \epsilon _{1}}+e^{-\beta \epsilon
_{2}}\right) }\allowbreak \allowbreak .\left( \text{we require chemical
potential should take such value that gives }\overline{N}=N\right) $

从上式解得$e^{-\alpha }=\frac{\overline{N}}{%
\overline{N}+1}\left( 1+e^{-\beta \epsilon _{1}}+e^{-\beta \epsilon
_{2}}\right) ^{-1}.$由巨正则分布的%
热力学公式知

$E_{G}=-\left( \frac{\partial \ln \Xi }{\partial \beta }\right) _{V,\alpha
},E_{G}=\frac{\partial \ln \left( 1-e^{-\alpha }\left( 1+e^{-\beta \epsilon
_{1}}+e^{-\beta \epsilon _{2}}\right) \right) }{\partial \beta }=\allowbreak 
\frac{e^{-\alpha }\left( \epsilon _{1}e^{-\beta \epsilon _{1}}+\epsilon
_{2}e^{-\beta \epsilon _{2}}\right) }{1-e^{-\alpha }\left( 1+e^{-\beta
\epsilon _{1}}+e^{-\beta \epsilon _{2}}\right) }\allowbreak \allowbreak =%
\frac{\overline{N}\left( \epsilon _{1}e^{-\beta \epsilon _{1}}+\epsilon
_{2}e^{-\beta \epsilon _{2}}\right) }{\left( 1+e^{-\beta \epsilon
_{1}}+e^{-\beta \epsilon _{2}}\right) },$

If we calculate $E_{C}$ from canonical ensemble formula $E_{C}=-\frac{%
\partial \ln Z}{\partial \beta },$we get the same result,i.e.$%
E_{G}\allowbreak =E_{C}.$

$S_{G}=k\left( \beta E_{G}+\alpha \overline{N}+\ln \Xi \right) =k\overline{N}%
\left( \frac{\beta \left( \epsilon _{1}e^{-\beta \epsilon _{1}}+\epsilon
_{2}e^{-\beta \epsilon _{2}}\right) }{1+e^{-\beta \epsilon _{1}}+e^{-\beta
\epsilon _{2}}}+\ln \frac{\overline{N}+1}{\overline{N}}\left( 1+e^{-\beta
\epsilon _{1}}+e^{-\beta \epsilon _{2}}\right) \right) +k\ln \left( 
\overline{N}+1\right) $

$S_{G}-S=k\overline{N}\left( \frac{\beta \left( \epsilon _{1}e^{-\beta
\epsilon _{1}}+\epsilon _{2}e^{-\beta \epsilon _{2}}\right) }{1+e^{-\beta
\epsilon _{1}}+e^{-\beta \epsilon _{2}}}+\ln \frac{\overline{N}+1}{\overline{%
N}}\left( 1+e^{-\beta \epsilon _{1}}+e^{-\beta \epsilon _{2}}\right) \right)
+k\ln \left( \overline{N}+1\right) $

-$Nk\left[ \ln \left( 1+e^{-\beta \epsilon _{1}}+e^{-\beta \epsilon
_{2}}\right) +\frac{\beta \epsilon _{1}e^{-\beta \epsilon _{1}}+\beta
\epsilon _{2}e^{-\beta \epsilon _{2}}}{1+e^{-\beta \epsilon _{1}}+e^{-\beta
\epsilon _{2}}}\right] =k\overline{N}\ln \frac{\overline{N}+1}{\overline{N}}%
+k\ln \left( \overline{N}+1\right) $,which is small compared with $\ln N$

if N is very large.


玻色子的有:12C atom, 4He atom, alpha
particle, H2 molecule

费米子的有:13C atom, H- ion(proton),3He
atom, 6Li- ion, e+


1.设系统的两个能级分%
别为$\epsilon _{1},\epsilon _{2},$系统按%
能量的分布为以下5种%
情况,$\left( \epsilon _{1},\epsilon _{2}\right) =\left(
4,0\right) ,\left( 3,1\right) ,\left( 2,2\right) ,\left( 1,3\right) ,\left(
0,4\right) $

其分别对应的微观状%
态数分别为1,16,36,16,1.$\left( \epsilon
_{1},\epsilon _{2}\right) =\left( 2,2\right) $出现的%
概率最大,因为其对应%
的微观状态数最多.

\begin{tabular}{llllll}
$\left( \epsilon _{1},\epsilon _{2}\right) $ & $\left( 4,0\right) $ & $%
\left( 3,1\right) $ & $\left( 2,2\right) $ & $\left( 1,3\right) $ & $\left(
0,4\right) $ \\ 
$P$ & $\frac{1}{70}$ & $\frac{8}{35}$ & $\frac{18}{35}$ & $\frac{8}{35}$ & $%
\frac{1}{70}$%
\end{tabular}

2.若对于玻色子,则以上5%
种宏观态对应的微观%
状态数分别为$\binom{7}{4}=\allowbreak 35$%
,80,100,80,35.$\left( \epsilon _{1},\epsilon _{2}\right) =\left( 2,2\right) $%
出现的概率最大.


$\left( 1\right) $由半经典分布公%
式知处于$\epsilon _{i}$能级的%
粒子数为$a_{i}=\omega _{i}e^{-\alpha -\beta \epsilon
_{i}},$故$\frac{a_{1}}{a_{0}}=e^{-\beta hv}$

$=e^{-\frac{hv}{kT}}=\exp \left( -\frac{0.3\times 1.6\times 10^{-19}}{%
1.38\times 10^{-23}\times 1000}\right) =\allowbreak 3.\,\allowbreak
086\,1\%. $

$\bigskip $

$\left( 2\right) $同上题$\frac{a_{1}}{a_{0}}=3e^{-\beta
\left( \epsilon _{1}-\epsilon _{0}\right) }=3\times \exp \left( -\frac{%
19.82\times 1.6\times 10^{-19}}{1.38\times 10^{-23}\times 1000}\right)
=\allowbreak 4.\,\allowbreak 758\,9\times 10^{-100}.$


\end{document}
