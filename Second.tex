
\documentclass{ctexart}
%%%%%%%%%%%%%%%%%%%%%%%%%%%%%%%%%%%%%%%%%%%%%%%%%%%%%%%%%%%%%%%%%%%%%%%%%%%%%%%%%%%%%%%%%%%%%%%%%%%%%%%%%%%%%%%%%%%%%%%%%%%%%%%%%%%%%%%%%%%%%%%%%%%%%%%%%%%%%%%%%%%%%%%%%%%%%%%%%%%%%%%%%%%%%%%%%%%%%%%%%%%%%%%%%%%%%%%%%%%%%%%%%%%%%%%%%%%%%%%%%%%%%%%%%%%%
\usepackage{amsmath}


\setcounter{MaxMatrixCols}{10}
%TCIDATA{OutputFilter=LATEX.DLL}
%TCIDATA{Version=5.00.0.2552}
%TCIDATA{<META NAME="SaveForMode" CONTENT="1">}
%TCIDATA{Created=Sunday, October 04, 2015 18:52:24}
%TCIDATA{LastRevised=Sunday, October 11, 2015 01:57:38}
%TCIDATA{<META NAME="GraphicsSave" CONTENT="32">}
%TCIDATA{<META NAME="DocumentShell" CONTENT="Scientific Notebook\Blank Document">}
%TCIDATA{CSTFile=Math with theorems suppressed.cst}
%TCIDATA{PageSetup=72,72,72,72,0}
%TCIDATA{AllPages=
%F=36,\PARA{038<p type="texpara" tag="Body Text" >\hfill \thepage}
%}


\newtheorem{theorem}{Theorem}
\newtheorem{acknowledgement}[theorem]{Acknowledgement}
\newtheorem{algorithm}[theorem]{Algorithm}
\newtheorem{axiom}[theorem]{Axiom}
\newtheorem{case}[theorem]{Case}
\newtheorem{claim}[theorem]{Claim}
\newtheorem{conclusion}[theorem]{Conclusion}
\newtheorem{condition}[theorem]{Condition}
\newtheorem{conjecture}[theorem]{Conjecture}
\newtheorem{corollary}[theorem]{Corollary}
\newtheorem{criterion}[theorem]{Criterion}
\newtheorem{definition}[theorem]{Definition}
\newtheorem{example}[theorem]{Example}
\newtheorem{exercise}[theorem]{Exercise}
\newtheorem{lemma}[theorem]{Lemma}
\newtheorem{notation}[theorem]{Notation}
\newtheorem{problem}[theorem]{Problem}
\newtheorem{proposition}[theorem]{Proposition}
\newtheorem{remark}[theorem]{Remark}
\newtheorem{solution}[theorem]{Solution}
\newtheorem{summary}[theorem]{Summary}
\newenvironment{proof}[1][Proof]{\noindent\textbf{#1.} }{\ \rule{0.5em}{0.5em}}


\begin{document}



统力第二次作业\qquad 赵%
丰2013012178


设体系中粒子数为N,体%
积为V$,$体系能量$\leq E,$下推%
导微观状态数与V,E,N的关%
系$.$

$\Omega \propto \underset{H<E}{\int }d^{3N}qd^{3N}p=\underset{\underset{i=1}{%
\overset{3N}{\Sigma }}\frac{p_{i}^{2}}{2m}<E}{\int }d^{3N}qd^{3N}p=\left(
\idotsint d^{3N}q\right) \underset{\underset{i=1}{\overset{3N}{\Sigma }}%
\frac{p_{i}^{2}}{2m}<E}{\idotsint }d^{3N}p=V^{N}K\left( 3N\right) \left(
2mE\right) ^{3N/2},$K为只和3N有关的%
常数.

$\implies \Omega \propto V^{N}E^{3N/2}$


对于谐振子$,$体系的能%
量为 $H=\underset{i=1}{\overset{n}{\Sigma }}\frac{p_{i}^{2}}{2m}+%
\frac{m}{2}\omega ^{2}\underset{i=1}{\overset{n}{\Sigma }}q_{i}^{2}.$

$\underset{H<E}{\int }d^{3N}qd^{3N}p=\underset{\frac{1}{2m}\underset{i=1}{%
\overset{3N}{\Sigma }}p_{i}^{2}+\frac{m}{2}\omega ^{2}\underset{i=1}{\overset%
{3N}{\Sigma }}q_{i}^{2}<E}{\idotsint }d^{3N}qd^{3N}p=\left( 2mE\right)
^{3N/2}\left( \frac{2E}{m}\right) ^{3N/2}K\left( 6N\right) =\left( 2E\right)
^{3N}K\left( 6N\right) $

$\implies \Omega \propto E^{3N},$与体积无关


由理想气体状态方程%
,\qquad $\left( 1\right) W_{1}=\int PdV=\int_{V_{0}}^{2V_{0}}\frac{NkT}{V}%
dV=NkT\ln 2,$对外做功

$U_{1}=\frac{3}{2}NkT_{0}$ 保持不变\qquad 由%
热力学第一定律$,\Delta
U=-W+Q\left( \text{对外做功取正值}%
\right) \implies Q_{1}=W_{1}=NkT\ln 2$

$\left( 2\right) W_{2}=\int PdV=P\Delta V=PV_{0}.$体系末%
状态温度变为 $T_{2}=\frac{T_{1}\left(
2V_{0}\right) }{V_{0}}=2T_{1}$

$\Delta U=\frac{3}{2}Nk\Delta T=\frac{3}{2}NkT_{1}\implies
Q_{2}=W_{2}+\Delta U=PV_{0}+\frac{3}{2}NkT_{1}.$


\end{document}
