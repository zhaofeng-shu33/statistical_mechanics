
\documentclass{ctexart}
\usepackage{amsmath}


%%%%%%%%%%%%%%%%%%%%%%%%%%%%%%%%%%%%%%%%%%%%%%%%%%%%%%%%%%%%%%%%%%%%%%%%%%%%%%%%%%%%%%%%%%%%%%%%%%%%%%%%%%%%%%%%%%%%%%%%%%%%%%%%%%%%%%%%%%%%%%%%%%%%%%%%%%%%%%%%%%%%%%%%%%%%%%%%%%%%%%%%%%%%%%%%%%%%%%%%%%%%%%%%%%%%%%%%%%%%%%%%%%%%
%TCIDATA{OutputFilter=LATEX.DLL}
%TCIDATA{Version=5.00.0.2552}
%TCIDATA{<META NAME="SaveForMode" CONTENT="1">}
%TCIDATA{Created=Thursday, November 05, 2015 09:02:56}
%TCIDATA{LastRevised=Thursday, November 05, 2015 17:37:42}
%TCIDATA{<META NAME="GraphicsSave" CONTENT="32">}
%TCIDATA{<META NAME="DocumentShell" CONTENT="Standard LaTeX\Blank - Standard LaTeX Article">}
%TCIDATA{CSTFile=40 LaTeX article.cst}

\newtheorem{theorem}{Theorem}
\newtheorem{acknowledgement}[theorem]{Acknowledgement}
\newtheorem{algorithm}[theorem]{Algorithm}
\newtheorem{axiom}[theorem]{Axiom}
\newtheorem{case}[theorem]{Case}
\newtheorem{claim}[theorem]{Claim}
\newtheorem{conclusion}[theorem]{Conclusion}
\newtheorem{condition}[theorem]{Condition}
\newtheorem{conjecture}[theorem]{Conjecture}
\newtheorem{corollary}[theorem]{Corollary}
\newtheorem{criterion}[theorem]{Criterion}
\newtheorem{definition}[theorem]{Definition}
\newtheorem{example}[theorem]{Example}
\newtheorem{exercise}[theorem]{Exercise}
\newtheorem{lemma}[theorem]{Lemma}
\newtheorem{notation}[theorem]{Notation}
\newtheorem{problem}[theorem]{Problem}
\newtheorem{proposition}[theorem]{Proposition}
\newtheorem{remark}[theorem]{Remark}
\newtheorem{solution}[theorem]{Solution}
\newtheorem{summary}[theorem]{Summary}
\newenvironment{proof}[1][Proof]{\noindent\textbf{#1.} }{\ \rule{0.5em}{0.5em}}


\begin{document}



Week Eight 

4.11 For the given substance,$g=1+\frac{j\left( j+1\right) +s\left(
s+1\right) -l\left( l+1\right) }{2j(j+1)}=2.$ The partition function is $%
z\left( \beta ,H\right) =\frac{Sh\left[ a\left( j+1\right) \right] }{Sh\frac{%
a}{2}}=\frac{Sh\left[ \frac{9}{2}a\right] }{Sh\frac{a}{2}},$where

$a=\beta \mu _{0}\mu _{B}gH=2\beta \mu _{0}\mu _{B}H.$ At high temperature,
by Curie's Law, $\chi =\frac{C}{T},$where $C=\frac{1}{3}j\left( j+1\right)
\mu _{0}\left( \mu _{B}g\right) ^{2}\frac{n}{k}=21\mu _{0}\mu _{B}^{2}\frac{n%
}{k},$where n represents the particle number per unit volume.

\bigskip At low temperature, $m=\frac{m}{H}=\frac{n\mu _{B}gj}{H}=\frac{7\mu
_{B}n}{H},$where $H$ represents the external field.

4.12 $\left( 1\right) $由题目知银原%
子的自旋磁矩只能取%
平行与反平行于磁场%
方向两种情形,记每一%
自旋磁矩大小为$\mu _{s}$则%
相应的磁势能为

$\mp \mu _{s}B$,采用正则系综的%
办法,单粒子partition function 为$%
z=e^{\beta \mu _{s}B}+e^{-\beta \mu _{s}B}=2\cosh \left( \beta \mu
_{s}B\right) \implies $设共有N个原子,%
其中$N_{+}$平行磁场场方%
向,另外$\left( N-N_{+}\right) $与磁场%
方向反平行,体系的总%
能量为$E=N_{+}\left( -\mu _{s}B\right) +\left(
N-N_{+}\right) \left( \mu _{s}B\right) .$

另一方面,$E=-N\frac{\partial \ln z}{\partial \beta }%
=-\mu _{s}BN\tanh \left( \beta \mu _{s}B\right) $.联立解%
得$N_{+}=\frac{1+\tanh \left( \beta \mu _{s}B\right) }{2}N.$

这一关系也可由classical Boltzmann 
统计给出,由于单粒子%
态简并度均为1由公式$%
a_{i}=e^{-\alpha -\beta \epsilon }$和$\frac{e^{-\alpha }}{N}=\frac{1}{z%
}$的换算关系可求出$N_{+}=%
\frac{N}{z}e^{\beta \mu _{s}B}=\frac{Ne^{\beta \mu _{s}B}}{2\cosh \left(
\beta \mu _{s}B\right) }$,由双曲函数的%
关系式可证明两式相%
同.

$\left( 2\right) $单分子平均磁矩%
为$\bar{m}=\mu _{s}N_{+}-\mu _{s}\left( N-N_{+}\right) =\mu _{s}\tanh
\left( \beta \mu _{s}B\right) ,$此式也可由"%
物态方程"推出:$M=\sum \mu _{i}e^{-\alpha
-\beta \epsilon _{i}}$

$=\frac{N}{z}\sum \mu _{i}e^{-\beta \epsilon _{i}}\implies \bar{m}=\frac{1}{z%
}\sum \mu _{i}e^{-\beta \epsilon _{i}}=\frac{1}{\beta }\frac{\partial \ln z}{%
\partial B}\implies \bar{m}=\mu _{s}\tanh \left( \beta \mu _{s}B\right) .$

5.1 题中所示为单原子理%
想玻色气体或费米气%
体,在弱简并即$e^{-\alpha }<1$的%
条件下由体系巨配分%
函数的对数为

$\ln \Xi =\underset{i}{\sum }\omega _{i}\ln \left( 1+e^{-\alpha -\beta
\epsilon _{i}}\right) ,$在准经典近似%
下$\ln \Xi $可由积分近似计%
算

$\ln \Xi =\int \frac{1}{h^{3}}\ln \left( 1+e^{-\alpha -\beta \epsilon
_{i}}\right) d\omega ,$积分限为状态%
空间$\left( \vec{p},\vec{q}\right) $,通过积%
分变换可将此积分变%
到能量空间上

$\ln \Xi =\int_{0}^{\infty }g\left( \epsilon \right) \ln \left( 1+e^{-\alpha
-\beta \epsilon }\right) d\epsilon ,$其中$g\left( \epsilon
\right) $表示体系能态密度%
即$g\left( \epsilon \right) d\epsilon $表示在$%
\epsilon \symbol{126}\epsilon +d\epsilon $能量区间%
内的微观状态数,对单%
原子理想气体可以求%
出$g\left( \epsilon \right) =\frac{1}{h^{3}}\frac{d\Omega \left(
\epsilon \right) }{d\epsilon }=\frac{2\pi V}{h^{3}}\left( 2m\right) ^{3/2}%
\sqrt{\epsilon },$代入上式有

$\ln \Xi =\int_{0}^{\infty }\frac{2\pi V}{h^{3}}\left( 2m\right) ^{3/2}\sqrt{%
\epsilon }\ln \left( 1+e^{-\alpha -\beta \epsilon }\right) d\epsilon $

$\bar{E}=-\frac{\partial \ln \Xi }{\partial \beta }=\int_{0}^{\infty }\frac{%
2\pi V}{h^{3}}\left( 2m\right) ^{3/2}\sqrt{\epsilon }\frac{\epsilon
e^{-\alpha -\beta \epsilon }}{1+e^{-\alpha -\beta \epsilon }}d\epsilon $

分部积分有$\bar{E}=\frac{-\int_{0}^{\infty }%
\frac{2\pi V}{h^{3}}\left( 2m\right) ^{3/2}\ln \left( 1+e^{-\alpha -\beta
\epsilon }\right) d\epsilon ^{3/2}}{-\beta }=\frac{3}{2}\frac{V}{\beta }%
\int_{0}^{\infty }\frac{2\pi }{h^{3}}\left( 2m\right) ^{3/2}\sqrt{\epsilon }%
\ln \left( 1+e^{-\alpha -\beta \epsilon }\right) d\epsilon $

由"物态方程"有$P=\frac{1}{\beta }\frac{%
\partial \ln \Xi }{\partial V}=\int_{0}^{\infty }\frac{2\pi }{h^{3}}\left(
2m\right) ^{3/2}\sqrt{\epsilon }\ln \left( 1+e^{-\alpha -\beta \epsilon
}\right) d\epsilon $

$\implies \bar{E}=\frac{3V}{2}P\implies P=\frac{2\bar{E}}{3V}.$


\end{document}
